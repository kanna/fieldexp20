\section*{Introduction}

In \proj we propose to close the observe-assimilate-predict-sample
loop in the coastal ocean. Model skill, especially in the coastal
ocean, is highly dependent on sparsely sampled in-situ measurements
and remote sensing data (when available). Typically ship-board
measurements, augmented by some fixed assets such as buoys, or
Lagrangian floats constitute the bulk of the observations. Quite often
these are not assimilated or if so, then frequently leading to poor
predictive skill.

Our objective in \proj is to demonstrate the applicability of
adaptively controlled marine robots in the aerial, surface and
underwater domains, while sampling the upper water-column 'at the
right place and time' driven by ocean models with increasing
predictive skill. Fig. \ref{fig:block-diag} shows a conceptual view of
the proposed field experiment.

\begin{figure}[!b]
  \centering
  \includegraphics[scale=0.15]{fig/ensemble.jpg}
  \caption{\proj will integrate ocean models with adaptive robotic vehicles
    in the coastal ocean, to increase model skill while increasing
    model accuracy and prediction with a tight loop.}
  \label{fig:block-diag}
\end{figure}

\section*{Motivation}

\proj in an inter-disciplinary field driven experiment spanning the
fields of Physical and Biological Oceanography, Artificial
Intelligence (including Machine Learning), autonomous systems and
Spatial Statistics. It will bring together researchers from these
diverse fields (all of whom have worked with one another) to build on
the integrative aspects of ocean observation especially in a dynamic
coastal environment.  In doing so, the tight integration between model
prediction and assimilation will be enhanced so as to provide
realistic forecasts of a range of bio-physical variables including
temperature, salinity, wind, surface and sub-surface currents. These
in turn will be used to \emph{intelligently target sampling} with
robotic vehicles in the air, surface and underwater domains. Important
outcomes of this proposed project include:

\begin{itemize}[noitemsep,topsep=5pt,parsep=0pt,partopsep=0pt,leftmargin=0.5cm]

\item rapid assessment of environmental state using state of the art
  methods in modeling, control and sampling with minimal human intervention
\item increasing model prediction with targeted sampling
\item real-time decision support to determine appropriate mix of
  robotic or manned assets (e.g. small boats or research vessel) for
  targeted sampling
\item demonstration of coordinated observations with an ensemble of
  robotic vehicles
\end{itemize}

The novelty is in the integrative aspects of this field exercise;
while individual aspects might have been demonstrated in small
experiments (including by members of this team), pulled together this
experiment will leap-frog experimental design, autonomous operations,
assimilation, modeling and prediction in ways not done before.

\section*{Location}

\begin{wrapfigure}{!h}{3.5in}
  \vspace{-0.5cm}
  \centering
  \includegraphics[scale=2]{fig/dom-carlos.jpg}
  \caption{A 70m research vessel, the NRP \emph{Dom Carlos} or its
    sister vessel will be available for 1 week for \proje.}
  \vspace{-0.3cm}
 \label{fig:vessel}
\end{wrapfigure}


We propose to hold the field experiment off of the western coast of
Portugal near the Nazar\'e canyon-Berlengas area for a number of
important reasons. The region offers one of the longest and deepest
canyon's in continental Europe with powerful tidal pumping, generation
of internal waves, seasonal wind forcing promoting canyon driven
upwelling and large upwelling filaments extending offshore for
hundreds of kilometers from shore with dispersal along the south and
the northern coasts as well as into the Atlantic.  This long canyon
and the presence of two islands just offshore provide an interesting
range of bio-geophysical phenomenon not observed elsewhere including
routine 30 meter waves in the surf zone; as a result Nazar\'e is
renowned as a surfing location
~\footnote{\url{https://www.youtube.com/watch?v=GJc4Ir78KdE}, and
  \url{https://www.youtube.com/watch?v=74pnrYPozcU}} with substantial
media coverage especially in the English speaking world.  Equally, the
Portuguese Hydrografic Institute
(\inste)~\footnote{\url{https://www.hidrografico.pt/}} maintains two
real-time multi-parameter buoys as well as tide gauges in the vicinity
of the canyon. We propose to leverage \inste's twice yearly
maintenance visits by their research vessels (Fig. \ref{fig:vessel})
to augment our logistical support and therefore propose to hold the
experiment in the September-October 2021 timeframe. If planned in
advance, the vessels can be made available for \proj for a duration of
1 week. See Fig. \ref{fig:studyarea-1} and \ref{fig:studyarea-2}.

\begin{figure}[!h]
  \vspace{-0.5cm} \centering \subfigure[Map of Portugal and the study
  area highlighted with the red
  rectangle.]{\label{fig:po-map}\includegraphics[scale=0.5]{fig/po-map.jpeg}}
  \hspace{+0.3cm} \subfigure[Zoomed in bathymetry showing the Nazar\'e
  canyon-Berlengas area (isobaths with depth in meters) and its
  environment including the placement of the two
  buoys.]{\label{fig:domain}\includegraphics[scale=0.5]{fig/domain.jpg}}
  \subfigure[Predictions of currents and temperature at 10m depth from
  a \texttt{HOPS} (Harvard Ocean Prediction System) model with
  assimilation of temperature and salinity measurements collected by
  CTD profiles and
  current-meters.]{\label{fig:model}\includegraphics[scale=0.50]{fig/model.jpeg}}
  \caption{\subref{fig:po-map} \& \subref{fig:domain} show detailed
    views of the proposed study area for \proje, while
    \subref{fig:model} shows model predictions for the dynamic region
    driven by bathymetry and external forcing.}
  \label{fig:studyarea-1}
\end{figure}


\begin{figure}[!h]
  \vspace{-0.5cm}
  \centering
  \subfigure[]{\label{fig:chlshelf}\includegraphics[scale=0.45]{fig/chl-shelf.png}}
  \subfigure[]{\label{fig:chldomain}\includegraphics[scale=0.4]{fig/chl-domain.png}}
  \subfigure{\includegraphics[scale=0.3]{fig/legend.png}}
  % \hspace{+0.3cm} 
  \subfigure[]{\label{fig:tmpchange}\includegraphics[scale=0.285]{fig/temp-change.png}}
  \subfigure[]{\label{fig:chlchange}\includegraphics[scale=0.285]{fig/chl-change.png}}
  \caption{300m resolution remote sensing images from Sentinel 3 from
    September 6\textsuperscript{th} 2019 showing along shelf
    \subref{fig:chlshelf} and zoomed in Chlrophyll filaments
    \subref{fig:chldomain} emanating from the proposed study area of
    Nazar\'e canyon-Berlengas. \subref{fig:tmpchange} and
    \subref{fig:chlchange} show time series with rapid changes of
    temperature and chlorophyll respectively in the area.}
  \label{fig:studyarea-2}
\end{figure}

\section*{Logistical and budgetary information}

Off shore in the Nazar\'e canyon-Berlengas area are two islands of
Berlengas and Farilho\~es part of a protected nature preserve. With
special permission we expect to be able to be based in Berlengas for a
period of 3 weeks to conduct off shore operations. A small team will
also be present onboard the \inst research vessel where if needed,
deployment and recovery of assets well off shore at the boundaries of
the survey region, can be done. In addition \inst will be in a
position to provide a RHIB or a rigid boat for near-shore operations
from the two islands.

\univ will provide the bulk of the robotic assets including
aerial, surface and underwater vehicles as also the team to operate
them. The team will be split between being based in Berlengas, the
research vessel and also provide remote monitoring from Porto. \soc
will provide at least one glider to operate outside the shelf to
augment model observations in the meso-scale.

Our approximate leveraged budget for this exercise is expected to be
\textbf{$\sim \$100$ K to support this 21 day (3 week)}
operation. This includes all robotic vehicles, the ship time and
personnel from \inste, faculty and staff from all the partner
organizations, transportation of assets and people in Portugal as also
insurance and communication costs for and during the exercise. The
budget will direcly support software development to extend observation
quality control, assimilation and ML based approaches to entropy
reduction in modeling, all based out of \unive.

\paragraph{Roles and responsibilities} \univ will be the lead
organization, provide aerial, surface and underwater vehicles,
communication equipment and command/control software while
coordinating all activities. Funding requested will primarily support
students and staff in Porto for software development and operations.
\inst will provide the assimilation, modeling and prediction with
shore-side models. In addition \inst will conduct CTD and vessel
mounted observations onboard the research vessel and also provide
access to their research vessel as well as a RHIB or rigid boat. \mit
will support \inst for \texttt{HOPS} modeling and augmentation for ML
capabilities. \colo and \ave will work on making biological
measurements and providing analysis and data to augment \inst
modeling. Table \ref{tab:roles} summarizes the roles and contributions
of all partners.

\begin{table}[!t]
  \centering
  \vspace{-0.5cm}
  \begin{tabular}{|p{1.5cm}|p{10cm}|p{4cm}|}\hline 
    \rowcolor{Gray}
    \bfseries Inst. &\bfseries Role &\bfseries Contributions\\
    \hline
    \org & Co-Lead organization, reporting, experiment design, outreach&\\
    \hline
    \univ & Co-Lead organization, CONOPS (concept of operations) design and
            implementation, software engineering
                                    &aerial/surface/underwater
                                      vehicles, comms, operations team\\
    \hline
    \inst & Physical Oceanography, Modeling, observation assimilation, prediction&\inst
                                                            research
                                                            vessel and
                                                            RHIB/rigid
                                                            boats,
                                                            local outreach\\
    \hline
    \ave & Biological sampling, lab analysis&\\
    \hline
    \soc & Physical Oceanography, glider operations& Gliders\\
    \hline
    \colo & Biological Oceanography, sampling algorithms&\\
    \hline
    \mit & Modeling, Machine Learning and entropy reduction&\\
    \hline
  \end{tabular}
  \caption{Roles and responsibilities and in-kind contributions in
    \proj for the proposed 2021 Sept-Oct field experiment. All team
    members will be collaboratively involved in experiment design and
    post experiment publishing.}
  \label{tab:roles}
\end{table}

\paragraph{Operational Considerations} While most experiments in the
coastal zone rely on ship-board measurements and gliders, the
challenging nature of this high-energy environment requires a mix of
assets in addition to gliders. Currents can be in the region of
$\sim 2$ knots, there is substantial variability in the upper
water-column and with high primary productivity and the existence of
coastal upwelling to provide for a rich nutrient base and the study
region has a robust presence of fishermen, nets and crab pots. We
expect to field an ensemble of low cost robotic vehicles including
propelled AUVs, unmanned surface vehicles and aerial vehicles with RGB
and potentially hyper-spectral imaging sensors. Accessibility to the
two islands will also provide some shelter from weather for continuous
operations including rapid launch/recovery of in-situ assets. In
addition, the PIs are in negotiations with \nas for one or more
overpasses of the \textbf{SeaHawk} nano-satellite with a
multi-spectral
imager~\footnote{\url{https://uncw.edu/socon/index.html}}. Finally,
\univ and \inst have a number of low cost temperature/density sensors
which when calibrated with CTDs on the vessel and robotic vehicles,
will enable local fisherman to provide timely data in this meso-scale
region. All such data will be assimilated into the ocean models run by
\inst and supported by \mit which continues to support \texttt{HOPS}
development.

\paragraph{COVID protocol} As per standard oceanographic cruise
protocol in current circumstances, we will follow the guidance of WHO
and the US CDC; all personnel will isolate 5 days before the
experiment after appropriate PCR tests for safety and stay in one
pod. Those onboard the \inst research vessel will follow Portuguese
Navy guidelines and also isolate prior to the cruise and have no
contact with shore based personnel during the course of the
experiment.

\section*{Outreach}

\inst has had continued and extensive engagement over the years with
the local authorities including the Nazar\'e city officials and local
fishermen. As a result we will engage the fishing community to carry
our low cost temperature/density sensors and also help provide access
to the team to be able to commute to the islands when necessary,
subject to health protocols. In addition, we expect to engage local
high school students prior and after the experiment, to entice them
into thinking about careers in science and technology, much as we have
done in the past with success in another ONR funded
program~\footnote{\url{https://sunfish.lsts.pt/en/outreach/education}}.

With the world-wide interest in the area \proj will also work with the
local authorities and \inst to showcase our work in Portuguese and
English language media outlets.

Finally, publications resulting from the project will be targeted
towards peer-reviewed journals including \emph{Science Robotics,
  J. Field Robotics, Int. J. of Robotics Research, AI Journal,
  Autonomous Robots, Oceanography, PLOSone, Frontiers} and high-impact
conferences such as \emph{AAAI, IJCAI, ISER, ICAPS, RSS, IROS, ICRA
  and IEEE AUV} where we have frequently published.

